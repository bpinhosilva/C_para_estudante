\chapter{Estruturas de Repetição}
Vamos imaginar uma situação na qual é preciso executar uma lista de tarefas. Suponha que tal lista é composta de 50 ítens. Nós poderíamos fazer um programa para executar as operações uma a uma, certo? Mas você concorda que isso seria algo tedioso caso a nossa lista mudasse para não mais 50, mas 250 tarefas? Bem, foi pensando em tais situações que os comandos de repetição foram inventados. São comumente chamados de laços de repetição (\textit{loops} no inglês). A ideia é repetir a execução de dadas tarefas baseado em uma condição. Assim como o nosso comando de repetição visto no capítulo anterior.

Vamos para um exemplo. É muito comum criarmos programas que ``contam'' os números. Como que faríamos para contar os dez primeiro números inteiros positivos? Intuitivamente faríamos dez ``\textit{printfs}''. 