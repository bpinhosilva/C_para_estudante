
\chapter{Estuturas de Contole}

Neste capítulo, vamos abordar as estruturas de controle. Bem, tais estruturas são responsáveis em decidir o fluxo de execução do programa. Suponha que você possua apenas R\$ \ 30,00 para comprar comida no mercado, mas você não sabe o preço exato de cada produto. A sua lista é composta de carne, arroz e feijão. Você precisa garantir que a carne seja comprada, \emph{\textbf{se}} sobrar dinheiro, você compra os outros produtos. Este tipo de estrutura é bem comum no nosso dia-a-dia. 

\section{Comando If}
O comando \textit{if}, como o próprio nome sugere, é responsável em tomar uma decisão caso alguma condição seja satisfeita. Logo abaixo existe a sintaxe desse comando:


\begin{lstlisting}[label=comandoIf,caption=Comando if]
// comando if
if ( <condicao> ) {
  <comandos>;
}
\end{lstlisting}

Na linguagem C, qualquer expressão que pode ser avaliada como valor verdadeiro ou falso pode ser inserida na seção de <condicao>. Aqui aplicam-se também todos os operadores relacionais da linguagem como $<, >, <=, !, ||,  \&\&$, etc.
Vamos agora ``trazudir'' a ação de ir ao mercado e comprar os produtos para programação.


\begin{lstlisting}[label=comandoIf,caption=Comando if]
#include <stdio.h>

#define PRECO_CARNE  15.50
#define PRECO_ARROZ  4.00
#define PRECO_FEIJAO 4.50

int main () {
  float dinheiro = 30.00;
  
  if (PRECO_CARNE < dinheiro)
    dinheiro -= PRECO_CARNE;
  
  if (PRECO_ARROZ < dinheiro)
    dinheiro -= PRECO_ARROZ;
  
  if (PRECO_FEIJAO < dinheiro)
  dinheiro -= PRECO_FEIJAO;    
  
  return 0;
}
\end{lstlisting}

Observe que a compra, caracterizada pela subtração do valor do produto, só é efetuada se a condição for satisfeita, que no caso é ter dinheiro suficiente para comprar o produto.
Há somente um comando dentro do {\tt if}, porém, se quisermos executar uma lista deles, basta inserir um par de chaves \{ \} e colocar todos os comandos dentro. Isso é o chamada ``bloco'' de execução que agrega um conjunto de operações a serem realizadas. Geralmente usado quando temos mais do que uma instrução a ser executada em comandos de controle.

\begin{lstlisting}[label=comandoIf,caption=Exemplo múltiplos comandos]
if ( <condicao> ) {
  <comando1>;
  <comando2>;
  <comando3>;
  .
  .
  .
  <comandoN>;
}
\end{lstlisting}

\textbf{Exemplo}:
  Faça um programa que some dois valores e imprima o resultado se eles forem positivos.

\begin{lstlisting}[label=comandoIf,caption=Exemplo múltiplos comandos]
#include <stdio.h>

int main () {
  int a, b, soma;
  scanf("%d", &a);  
  scanf("%d", &b);
  
  if ( a > 0 && b > 0) {
    soma = a + b;
    printf ("%d\n", soma);
  }
  
  return 0;
}
\end{lstlisting}

\section{Else}
Até agora só poderíamos executar uma tarefa caso a condição fosse verdadeira, mas e no caso dela ser falsa? Existe o complemento do comando \textit{if} que é o \textit{else}. Else significa ``caso contrário'' e é utilizado para definir ações a serem tomadas quando a condição testada for falsa.
Por exemplo, suponha que queiramos tomar decisões em ambos casos, quando a condição for verdadeira e quando for falsa. Se num sistema de inscrição, por exemplo, alguém for fazer uma inscrição e houver vagas disponíveis, o sistema irá redirecionar a tela para o preenchimento de formulário, caso contrário, exibirá uma mensagem de alerta. Isso é a aplicação do comando \textit{if-else} para testar condições.

\begin{lstlisting}[label=comandoIf,caption=Exemplo com else]
if ( <numero_vagas_maior_que_zero> ) {
  <redireciona_pagina>;
  <exibe_formulario>;
}
else {
  <exibe_mensagem>;
  <aborta_inscricao>;
}
\end{lstlisting}

\newpage
\section{Exercícios}

\begin{enumerate}
  \item Elabore um programa que decida qual o maior de dois números inteiros e imprima na saída padrão a seguinte     
  frase: ``O maior número é: <valor>.'' Em que <valor> deve ser o maior número da entrada. Os dados de entrada são separados por espaço em branco.\\
  \\Exemplo de entrada: 12 56
  \\Exemplo de saída: O maior número é: 56.
  
  
  \item Considerando que as dimensões de uma academia são fornecidas pela entrada padrão, construa um programa que        
  decida se ela vai caber dentro da área de 400m$^{2}$ reservada para a construção de um espaço recreativo no       
  centro da sua cidade. Imprima na saída apenas uma string contendo Sim ou Não. (Não esqueça de imprimir caracter de nova linha);\\
  \\ Exemplo de entrada: \\10.45 18.98
  
  \item Considere o seguinte intervalo: $[0, 170)$. Dado um número inteiro qualquer, imprima uma mensagem 
  indicando se ele pertence ou não ao intervalo dado.
  
  \item Conversão entre números inteiros é muito comum, usando base decimal ou binária. Sua amiga está aprendendo sobre eletrônica digital e ela está tendo dificuldades para converter de binário para decimal. Sua tarefa é fazer um programa para ajudá-la a conferir os resultados dela. Para isso, dada uma entrada consistindo de números com suas representações decimal e binária, construa um programa que imprima a mensagem na tela informando se a conversão está ``certa'' ou ``errada''. Os números são separados por espaço em branco e cada sequência de dois é separada por uma nova linha. Cada número binário possui 4 dígitos. Os casos de teste terminam quando um 0 é encontrado.\\
  
  
    \begin{minipage}{.45\textwidth}
      \centering

Exemplo de entrada: 

  1 0001
  
  2 0010
  
  3 0011
  
  4 1111  
  
  0
    \end{minipage}
    \begin{minipage}{.45\textwidth}
      \centering
Exemplo de saída:

  certo
  
  certo
  
  certo
  
  errado
    \end{minipage}

  \item Dado um número inteiro qualquer, faça um programa que verifica se ele é par ou ímpar. Imprima na saída padrão uma mensagem informando o resultado.
  
  \item Suponha que um carro está se movendo com uma velocidade de 90km/h na BR-324 sentido Salvador e ele ainda tem 197km para chegar à capital. Admita que ele mantenha essa velocidade constante em todo o trecho, faça um programa que calcule e verifique se ele chegará no seu destino em no máximo nas próximas 3 horas. Em caso afirmativo, imprima uma mensagem na tela informando ``É possível!'', caso contrário, ``Não é possível.'';
  
  \item Seu amigo está querendo fazer uma calculadora de operações lógicas e pediu sua ajuda. A calculadora possui duas operações que são: \emph{xor} (ou exclusivo) e \emph{and}. Para poder usar tal operação é necessário fazer a leitura de um caracter. O 'x' indica uma operação xor e um 'n' uma operação and. Desenvolva um programa que faça a leitura dessa caracter e imprima na tela o resultado da operação desejada entre os números 1 e 2, depois entre 2 e 3, e, por fim, entre 3 e 4. (Observação: converta os números para base binária e verifique o resultado obtido).
  
  \item Um palíndromo pode ser compreendido como uma sequência de caracteres ou números que podem ser lidos tanto da esquerda para a direita quanto da direita para a esquerda sem alterar o resultado, é como se ele fosse espelhado. Por exemplo: 121 é um palíndromo. Construa um programa que verifica se um número de 3 dígitos forma um palíndromo. Imprima na saída padrão o resultado ``sim'' ou ``não''.

  \item Dada uma sequência de ângulos em graus, verifique quais deles são agudos e quais são obtusos. A lista possui 10 número e são separados por caracter de nova linha. Para cada valor, imprima na tela a string informando ``agudo'' ou ``obtuso'' para o respectivo valor lido.
  
\end{enumerate}